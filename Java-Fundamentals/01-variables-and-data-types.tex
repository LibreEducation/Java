\documentclass[a4paper]{article}

% Use Package
\usepackage[english]{babel}
\usepackage[utf8]{inputenc}
\usepackage{amsmath}
\usepackage{graphicx}
\usepackage[colorinlistoftodos]{todonotes}
\usepackage{fontspec}
\usepackage{geometry}
\usepackage{layout}
\usepackage{titling}
\usepackage{listings}  

% Define Commands
\newcommand{\subtitle}[1]{%
  \posttitle{%
    \par\end{center}
    \begin{center}\large#1\end{center}
    \vskip0.5em}%
}

% Initial Set Up
\setmainfont{Arial}
\lstdefinestyle{codeStyle}{
  belowcaptionskip=1\baselineskip,
  breaklines=true,
  frame=L,
  xleftmargin=\parindent,
  language=Java,
  showstringspaces=false,
  basicstyle=\footnotesize\ttfamily,
  keywordstyle=\bfseries\color{green!40!black},
  commentstyle=\itshape\color{purple!40!black},
  identifierstyle=\color{blue},
  stringstyle=\color{orange},
}
\lstset{ 
  backgroundcolor=\color{white}, 
  breakatwhitespace=false,         
  breaklines=true,              
  captionpos=b,                    
  commentstyle=\color{green},    
  deletekeywords={...},            
  escapeinside={\%*}{*)},          
  extendedchars=true,              
  frame=single,	                 
  keepspaces=true,                
  keywordstyle=\color{blue},      
  language=Java,                  
  otherkeywords={*,...},           
  numbers=left,                    
  numbersep=5pt,                   
  numberstyle=\color{gray},
  rulecolor=\color{white},         
  showspaces=false,             
  showstringspaces=false,          
  showtabs=false,                  
  stepnumber=1,                    
  stringstyle=\color{mauve},     
  tabsize=2,	                   
  title=\lstname,                   
  escapechar=@,
  style=codeStyle
}

\title{Variables and Data Types}
\subtitle{Java Fundamentals}
\date{} 
\author{Libre Education}
\begin{document}
\maketitle


\begin{figure}[b]
\includegraphics{BY-SA}
\centering
\end{figure}

\newpage

\section*{Variables}

A variable is place set aside in your computer�s memory to store a data value. The variable has a data type assigned to it to show what kind of data is being stored in the variable.

\section*{Data Types}

Data types represent the type of data that is being stored in a variable. They can be defined by the user or built into the language, these are called primitive data types. 
\\\\
e.g. String is a data type that stores a string of characters, for example a sentence.

\subsection*{Primitive Data Types}
A primitive data type is a data type that is predefined or �built-into� the language. There are eight primitive data types in Java:
\begin{enumerate}

\item \textbf{int} - Stores integers. Minimum value of \(-2^{31}\). Maximum value of \(2^{31}-1\). Default value is 0

\item \textbf{double} - Stores decimals. Minimum value of \(2^{-1074}\). Maximum value of (2 to \(2^{-52})�2^{1023}\). Default value is 0.0d

\item \textbf{char} - Stores all unicode characters. Minimum value of 0 ('\textbackslash u0000'). Maximum value of 65,535 ('\textbackslash uffff'). Default value is 0 ('\textbackslash u0000')

\item \textbf{boolean} - Stores either a �true� or �false� value. Represents 1 bit of information. Default value is �false�

\item \textbf{long} - Stores a wider range of integers than int. Minimum value of -263. Maximum value of \(263^{-1}\). Default value is 0L

\item \textbf{short} - Stores a smaller range integers than int. Minimum value of -215. Maximum value of 215-1. Default value is 0

\item \textbf{byte} - Stores 1 byte of data. Minimum value of \(-128^{1}\). Maximum value of 127. Default value is 0

\item \textbf{float} - Stores floating point decimal. Minimum value of \( 2^{-149}\). Maximum value of (2 to \(2^{-23})�2^{127}\). Default value is 0f

\end{enumerate}

\newpage

\section*{Declaring a Variable}

\subsection*{Syntax}
\begin{lstlisting}
data_type variable_name = value;
\end{lstlisting}

\subsubsection*{Example}
\begin{lstlisting}
public class Main {

   public static void main(String[] args) {

       String st = "Hello World";
       System.out.println(st);

       //Eight Primitive Data Types
       int i = 62;

       double d = 3.14159d;
       double d2 = 4.20341;

       char c1 = 'A';
       char c2 = 65;
       char c3 = '\u0041';

       boolean bo = true;

       long L = 68364278382738L;

       short sh = 82;

       byte b1 = 3;
       byte b2 = 'A';

       float f = 2.4f;

   }
}
\end{lstlisting}

\newpage

\section*{Other Resources}
\begin{enumerate}
\item The Java Tutorials (docs.oracle.com/javase/tutorial/java/nutsandbolts/datatypes.html)
\item TutorialsPoint (www.tutorialspoint.com/java/java\_basic\_datatypes.htm)
\item Java For Dummies (www.dummies.com/how-to/content/the-eight-data-types-of-java.html)
\item Java Programming - WikiBooks (en.wikibooks.org/wiki/Java\_Programming/Primitive\_Types)
\item w3Resource (www.w3resource.com/java-tutorial/java-premitive-data-type.php)
\item CMU (www.cs.cmu.edu/~mrmiller/15-110/Handouts/primitiveData.pdf)
\item TutorialsCollection (www.tutorialscollection.com/java-data-types-primitive-data-types-in-java)
\item ZetCode (zetcode.com/lang/java/datatypes)
\end{enumerate}


\end{document}