\documentclass[a4paper]{article}

% Packages
\usepackage[english]{babel}
\usepackage[utf8]{inputenc}
\usepackage{amsmath}
\usepackage{graphicx}
\usepackage[colorinlistoftodos]{todonotes}
\usepackage{helvet}
\usepackage{geometry}
\usepackage{layout}
\usepackage{titling}
\usepackage{listings}

% Commands
\newcommand{\subtitle}[1]{%
  \posttitle{%
    \par\end{center}
    \begin{center}\large#1\end{center}
    \vskip0.5em}%
}

\renewcommand{\familydefault}{\sfdefault}

% Set Up
\lstdefinestyle{codeStyle}{
  belowcaptionskip=1\baselineskip,
  breaklines=true,
  frame=L,
  xleftmargin=\parindent,
  language=Java,
  showstringspaces=false,
  basicstyle=\footnotesize\ttfamily,
  keywordstyle=\bfseries\color{green!40!black},
  commentstyle=\itshape\color{purple!40!black},
  identifierstyle=\color{blue},
  stringstyle=\color{orange},
}
\lstset{
  backgroundcolor=\color{white},
  breakatwhitespace=false,
  breaklines=true,
  captionpos=b,
  commentstyle=\color{green},
  deletekeywords={...},
  escapeinside={\%*}{*)},
  extendedchars=true,
  frame=single,
  keepspaces=true,
  keywordstyle=\color{blue},
  language=Java,
  otherkeywords={*,...},
  numbers=left,
  numbersep=5pt,
  numberstyle=\color{gray},
  rulecolor=\color{white},
  showspaces=false,
  showstringspaces=false,
  showtabs=false,
  stepnumber=1,
  stringstyle=\color{mauve},
  tabsize=2,
  title=\lstname,
  escapechar=@,
  style=codeStyle
}

% Document start
\title{Operators}
\subtitle{Java Fundamentals}
\date{}
\author{Libre Education}
\begin{document}
\maketitle


\begin{figure}[b]
\includegraphics{BY-SA}
\centering
\end{figure}

\newpage

\section*{Operators}

An operator is a construct that performs a function to one or more values.

\section*{Types of Operators}
Operators are grouped by what kind of operation they perform.

\subsection*{Arithmetic Operators}
Arithmetic operators perform typical mathematical functions to two values.
\begin{enumerate}

\item = (Simple Assignment) - Assigns the value on the right to the value on the
left

\item += (Add And Assign) - Adds the two values together and then assigns the
value on the left to the result

\item -= (Subtract And Assign) - Subtracts the right value from the left value
and then assigns the value on the left to the result

\item *= (Multiply And Assign) - Multiplies the values together and then assigns
the value on the left to the result

\item /= (Divide And Assign) - Divided the value on the left by the value on the
right and then assigns the value on the left to the result

\end{enumerate}

\subsection*{Unary Operators}
Unary operators require and perform a function to one value.

\begin{enumerate}

\item + (Unary Plus)

\item - (Unary Minus)

\item + + (Increment) - Increases the value by 1

\item - - (Decrement) - Decreases the value by 1

\item ! (Logical Compliment) - Inverts the value of a boolean

\end{enumerate}

\subsection*{Equality And Relational Operators}
Equality and Relational operators check for a condition and return a boolean but
do not manipulate values
\begin{enumerate}

\item == (Equal To) - Returns true if the two values are equal to each other

\item != (Not Equal To) - Returns true if the two values are not equal to each
other

\item > (Greater Than) - Returns true if the value on the left is greater than
the value on the right

\item < (Less Than) - Returns true if the value on the left is less than the
value on the right

\item >= (Greater Than or Equal To) - Returns true if the value on the left is
greater than or equal to the value on the right

\item <= (Less Than or Equal To) - Returns true of the value on the left is less
than or equal to the value on the right

\item instanceof (Instance of) - Compares an object and a class. Returns true if
the object provided on the left is an instance of the class provided on the
right

\end{enumerate}

\subsection*{Conditional Operators}
Conditional operators check two statements or expressions and return a boolean
based on the result
\begin{enumerate}

\item \&\& (Conditional AND) - Returns true if both statements are true

\item || (Conditional OR) - Returns true if at least one of the statements are
true

\end{enumerate}

\subsection*{Bitwise and Bitshift Operators}
Bitwise and Bitshift operators manipulate values at the byte level. They take
place between each parallel pair of bits in the value
\begin{enumerate}

\item $\sim$ (Unary Bitwise Complement) - Inverts each bit in the value

\item \& (Bitwise AND) - If both corresponding bits are 1, the result is 1

\item | (Bitwise Inclusive OR) - If either of the corresponding bits are 1, the
result is 1

\item \string^ (Bitwise Exclusive OR) - If both of the corresponding bits are 1,
the result is 1

\end{enumerate}

\subsection*{Ternary Operator}
? : - The ternary operator provides a shorter way to write and if...then...else
statement and takes three operands: an expression and two values. If the
expression evaluates to true the first value is returned and if the expression
evaluates to false the second value is returned.

\subsubsection*{Syntax}
\begin{lstlisting}
result = expression ? value1 : value2;
\end{lstlisting}

\subsubsection*{Example}
\begin{lstlisting}
boolean expression = true;
String result = expression ? "The expression is true" : "The expression is
false";
System.out.println(result);
// Output: The expression is true

\end{lstlisting}


\newpage

\section*{Other Resources}
\begin{enumerate}

\item Wikipedia (en.wikipedia.org/wiki/Operator\_(computer\_programming))

\item The Java Tutorials (docs.oracle.com/javase/tutorial/java/nutsandbolts
/operators.html)

\item TutorialsPoint (www.tutorialspoint.com/java/java\_basic\_operators.htm) \\

Bitwise and Bitshift Operators

\item Rose India (www.roseindia.net/java/master-java/bitwise-bitshift-
operators.shtml)

\end{enumerate}


\end{document}
