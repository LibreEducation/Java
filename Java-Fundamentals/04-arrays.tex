\documentclass[a4paper]{article}

% Use Package
\usepackage[english]{babel}
\usepackage[utf8]{inputenc}
\usepackage{amsmath}
\usepackage{graphicx}
\usepackage[colorinlistoftodos]{todonotes}
\usepackage{fontspec}
\usepackage{geometry}
\usepackage{layout}
\usepackage{titling}
\usepackage{listings}  

% Define Commands
\newcommand{\subtitle}[1]{%
  \posttitle{%
    \par\end{center}
    \begin{center}\large#1\end{center}
    \vskip0.5em}%
}

% Initial Set Up
\setmainfont{Arial}
\lstdefinestyle{codeStyle}{
  belowcaptionskip=1\baselineskip,
  breaklines=true,
  frame=L,
  xleftmargin=\parindent,
  language=Java,
  showstringspaces=false,
  basicstyle=\footnotesize\ttfamily,
  keywordstyle=\bfseries\color{green!40!black},
  commentstyle=\itshape\color{purple!40!black},
  identifierstyle=\color{blue},
  stringstyle=\color{orange},
}
\lstset{ 
  backgroundcolor=\color{white}, 
  breakatwhitespace=false,         
  breaklines=true,              
  captionpos=b,                    
  commentstyle=\color{green},    
  deletekeywords={...},            
  escapeinside={\%*}{*)},          
  extendedchars=true,              
  frame=single,	                 
  keepspaces=true,                
  keywordstyle=\color{blue},      
  language=Java,                  
  otherkeywords={*,...},           
  numbers=left,                    
  numbersep=5pt,                   
  numberstyle=\color{gray},
  rulecolor=\color{white},         
  showspaces=false,             
  showstringspaces=false,          
  showtabs=false,                  
  stepnumber=1,                    
  stringstyle=\color{mauve},     
  tabsize=1,	                   
  title=\lstname,                   
  escapechar=@,
  style=codeStyle
}

\title{Arrays}
\subtitle{Java Fundamentals}
\date{} 
\author{Libre Education}
\begin{document}
\maketitle


\begin{figure}[b]
\includegraphics{BY-SA}
\centering
\end{figure}

\newpage

\section*{Arrays}

An array is a data structure that can hold multiple values of the same data type. It is therefore considered as a container object.

\section*{Declaring An Array}
When declaring an array, like any variable the data type has to be specified as well as the maximum number of values that the array can contain.

\subsubsection*{Syntax}
\begin{lstlisting}
data_type[] array_name = new data_type[number_of_values];
\end{lstlisting}

\subsubsection*{Example}
\begin{lstlisting}
int[] arr = new int[5]; // An array that can hold 5 int values
\end{lstlisting}

\section*{Indexes}
Each position in the array is called an index. Indexes differ from regular counting as they start from 0. So an array that can hold 5 values would have a starting index of 0 and an ending index of 4.

\section*{Assigning Values}

\subsection*{Assigning values to a specific index}

\subsubsection*{Syntax}
\begin{lstlisting}
array_name[index] = value;
\end{lstlisting}

\subsubsection*{Example}
\begin{lstlisting}
int[] arr = new int[5];
arr[0] = 4; // Assigning a value to the first (0) index
arr[1] = 8; // Assigning a value to the second (1) index
arr[2] = 7; // Assigning a value to the third (2) index
\end{lstlisting}

\subsection*{Assigning values at declaration}
When assigning values at declaration, the number of values assigned becomes the number of values the array can contain.

\subsubsection*{Syntax}
\begin{lstlisting}
data_type[] array_name = {value, value, value};
\end{lstlisting}

\subsubsection*{Example}
\begin{lstlisting}
int[] arr = {4, 8, 7 , 2, 3};
// An array that can hold 5 values, with 5 values assigned to each index
\end{lstlisting}

\section*{Accessing Values}
To access a specific value in an array, the value is referred to by stating the array name and the index at which the value is located.

\subsubsection*{Syntax}
\begin{lstlisting}
array_name[index];
\end{lstlisting}

\subsubsection*{Example}
\begin{lstlisting}
int[] arr = {4, 8, 7 , 2, 3};
System.out.println(arr[2]);
// Output: 7
\end{lstlisting}

\section*{Manipulating Arrays}
There are many manipulations that can be applied to arrays as well many methods (functions). The following is a list of the most common methods and manipulations made to arrays.

\subsection*{Convert an array to string}
\begin{lstlisting}
int[] arr = {4, 8, 7};
String stringArr = Arrays.toString(arr);
System.out.println(stringArr);
// Output: [4, 8, 7]
\end{lstlisting}

\subsection*{Sort in ascending order}
\begin{lstlisting}
int[] arr = {8, 2, 5, 1};
Arrays.sort(arr);
System.out.println(Arrays.toString(arr));
// Output: [1, 2, 5, 8]
\end{lstlisting}

\subsection*{Check if an array contains a value}
\begin{lstlisting}
String[] stringArr = {"a", "b", "c"};
boolean containsValue = Arrays.asList(stringArr).contains("a");
System.out.println(containsValue);
// Output: true
\end{lstlisting}

\section*{Multi-Dimensional Arrays}
Multi-dimensional arrays can be considered as an array of arrays where elements can themselves be an array.

\subsection*{Two dimensional arrays}
When using two dimensional arrays it can often be easier to think of the array as a grid where the first bracket specifies the row and the second bracket specifies the column thus giving the �coordinates� of a specific value.

\subsubsection*{Syntax}
\begin{lstlisting}
data_type[][] array_name = {1D_Array, 1D_Array, 1D_Array};
\end{lstlisting}

\subsubsection*{Example}
\begin{lstlisting}
int[][] twoDimArr = { {7, 4, 1} , 
    				{2, 5, 9, 3} };

System.out.println(twoDimArr[1][0]);
// Output: 2
System.out.println(Arrays.toString(twoDimArr[1]));
// Output: [2, 5, 9]
\end{lstlisting}

\subsubsection*{Note}
Using just one bracket will refer the whole array. E.g. array\_name[index]


\newpage

\section*{Other Resources}
\begin{enumerate}

\item The Java Tutorials (docs.oracle.com/javase/tutorial/java/nutsandbolts/arrays.html)

\item TutorialsPoint (www.tutorialspoint.com/java/java\_arrays.htm) 

\item Java Programming - WikiBooks (en.wikibooks.org/wiki/Java\_Programming/Arrays) 

\item Princeton (introcs.cs.princeton.edu/java/14array)

\item ZetCode (zetcode.com/lang/java/arrays)

\item w3Resource (www.w3resource.com/java-tutorial/java-arrays.php)

\end{enumerate}


\end{document}